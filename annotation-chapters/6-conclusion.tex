\chapter{Заключение}
\label{chap:conclusion}

\section{Резюме}

Мы разработали и реализовали начальный прототип языкового сервера и расширения для VS Code для экспериментального доказательного помощника \Rzk{}. Промежуточные результаты этой работы были представлены на школе \textit{Взаимодействие доказательных помощников и математики}\footnote{\url{https://itp-school-2023.github.io/}} в Германии в сентябре 2023 года. Это помогло собрать обратную связь от пользователей и реагировать на неё непосредственно. Мы уверены, что текущие пользователи в основном удовлетворены прототипом инструментов, но также предоставили полезные предложения для дальнейшего улучшения.

\section{Планы на будущее}

Текущая реализация языкового сервера для \Rzk{} находится на ранних стадиях. Есть множество отсутствующих функций и мест для улучшения. Например, необходимо провести формальное исследование эффективности продукта этой работы с помощью опроса пользователей для выявления наиболее важных функций для работы.

В будущем мы планируем поддержку отображения информации о переменных при наведении, переход к определению, переименование символов и другие полезные функции LSP. Также планируется поддержка отображения топов как изображений и добавление информационного веб-представления, аналогичного тому, которое предоставляет Lean 4 \cite{Nawrocki2023}.

Также ожидается, что из этого проекта вырастет библиотека на Haskell для разработки языковых серверов (особенно для доказательных помощников).
