\chapter{Реализация}
\label{chap:impl}

\section{Языковой сервер \Rzk{} и расширение для VS Code}

Описываем реализацию языкового сервера и расширения для \Rzk{} в среде VS Code. Языковой сервер имеет доступ к внутренним компонентам доказательного ассистента и предоставляет интерфейс, соответствующий протоколу языкового сервера. Расширение для VS Code действует как посредник между редактором и языковым сервером.

\subsection{Особенности}

Языковой сервер поддерживает интуитивный интерфейс и подсветку синтаксиса, соответствующую ожиданиям математиков и программистов. Пользователи получают ясную навигацию и подсветку синтаксиса, что улучшает читаемость кода и облегчает обнаружение ошибок.

\subsubsection{Автодополнение кода и предложения}

Расширение для VS Code \Rzk{} использует LSP для автодополнения кода и предложений, помогая писать код более эффективно и снижая вероятность ошибок.

\subsection{Расширение для VS Code}

Расширение управляет установкой языкового сервера на основных ОС, используя бинарные файлы с GitHub. Исходный код доступен на GitHub\footnote{\url{https://github.com/rzk-lang/vscode-rzk}}, а расширение на Visual Studio Marketplace\footnote{\url{https://marketplace.visualstudio.com/items?itemName=NikolaiKudasovfizruk.rzk-1-experimental-highlighting}} и Open VSX\footnote{\url{https://open-vsx.org/extension/NikolaiKudasovfizruk/rzk-1-experimental-highlighting}}.

\subsubsection{Установка и активация}

После активации расширение проверяет наличие \Rzk{} в \texttt{PATH}. Если исполняемый файл не найден, предлагается скачать его с GitHub. Пользователь может указать пользовательский путь к бинарному файлу.

\subsection{Языковой сервер \Rzk{}}

Языковой сервер, написанный на Haskell, поддерживает подсветку синтаксиса, диагностические сообщения и автодополнение текста. Он совместим с любыми редакторами, поддерживающими LSP.

\section{Вспомогательные инструменты}

Разработаны вспомогательные инструменты, такие как плагин для MkDocs и GitHub Action для использования \Rzk{} в CI-пайплайнах.

\subsection{Плагин MkDocs}

Плагин для MkDocs, доступный на PyPI\footnote{\url{https://pypi.org/project/mkdocs-plugin-rzk/}}, позволяет рендерить диаграммы в документации, используя \Rzk{}.

\subsection{GitHub Action}

GitHub Action, доступный на GitHub\footnote{\url{https://github.com/rzk-lang/rzk-action}}, позволяет проверять корректность доказательств в CI-пайплайнах, обеспечивая статус проверки в pull requests.
