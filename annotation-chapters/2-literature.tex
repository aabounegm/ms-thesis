\chapter{Обзор литературы}
\label{chap:lr}

Это исследование основывается на работах в областях ассистентов доказательств, языковых серверов и архитектуры компиляторов на основе запросов. В этой главе рассматривается наиболее релевантная литература.

\section{Ассистенты доказательств}

\Rzk{}~\cite{kudasov2023experimental} основывается на последних разработках других ассистентов доказательств, предоставляя удобный интерфейс для синтетических $\infty$-категорий.

\subsection{Ассистенты доказательств для синтетических $\infty$-категорий}

\subsubsection{Coq}

Coq \cite{huet1997coq} является одним из первых ассистентов доказательств и поддерживает зависимые типы и индуктивные конструкции. Он используется в различных областях, включая формальную верификацию и теорию программирования. Библиотека UniMath \cite{DanielGrayson2024, MacPherson2019} добавляет поддержку Homotopy Type Theory и унивальных оснований.

\subsubsection{Cubical Agda}

Agda \cite{BoveDybjerNorell2009} — это ассистент доказательств, основанный на парадигме "доказательства как типы". Cubical Agda \cite{VEZZOSI2021} расширяет Agda кубической теорией типов, позволяя формализовать Homotopy Type Theory \cite{Cohen2016}.

\subsection{Библиотеки для разработки ассистентов доказательств}

\subsubsection{Proof General}

Proof General \cite{Aspinall2000} — это инструмент для разработки интерактивных доказательных систем на основе редактора Emacs. Он поддерживает множество известных ассистентов доказательств, таких как Coq и Isabelle.

\subsubsection{A.L.G.A.E.}

A.L.G.A.E. \footnote{\url{https://redprl.org/\#algae}} — это проект, предоставляющий библиотеки для разработки ассистентов доказательств на OCaml. Он включает инструменты для диагностики компилятора, уровней вселенных и поддержки кубической теории типов \cite{Kovacs2021}.

\section{Протокол языкового сервера}

Протокол языкового сервера (LSP) был разработан Microsoft для отделения реализации языка от интерфейса редактора \cite{Buender2019}. Он используется многими инструментами разработки, такими как Visual Studio Code, Emacs и Vim.

\subsection{Specification Language Server Protocol}

LSP был разработан для языков общего назначения, но для спецификационных языков, таких как ассистенты доказательств, требуется расширение. Jonas Rask et al. \cite{JonasKjaerRask2021} предложили расширение LSP для спецификационных языков, называемое Specification Language Server Protocol (SLSP). Оно добавляет новые запросы и уведомления, а также вид для отображения информации о доказательствах.

\subsection{Расширение VS Code для Lean 4}

Основной способ разработки программ на Lean 4 — это использование расширения VS Code, которое предоставляет интерфейс для работы с Lean \cite{Nawrocki2023}. Это расширение использует LSP для связи с сервером Lean и предоставляет такие функции, как панель Info View.

\subsection{Отчеты о языковых серверах для ассистентов доказательств}

Несколько отчетов описывают разработку языковых серверов для ассистентов доказательств. Bour et al. \cite{Bour2018} описывают языковой сервер для OCaml, Kaliszyk \cite{Kaliszyk2007} предлагает архитектуру веб-интерфейсов для Coq, а Tavante \cite{Tavante2021} исследует инструменты для ассистентов доказательств.

\section{Архитектура компиляторов на основе запросов}

Традиционная компиляция состоит из пяти фаз \cite{dragon-book}: лексический анализ, синтаксический анализ, семантический анализ, генерация кода и оптимизация. Архитектура компиляторов на основе запросов \cite{ollef-rock} предназначена для интерактивных сред программирования, где каждая фаза представлена как независимый запрос. Эта архитектура более гибкая и позволяет кэшировать результаты запросов, что ускоряет интерактивные среды.

Недавние исследования показывают применение такой архитектуры в отладке объектно-ориентированных программ \cite{Lencevicius1997}. Lenkefi и Mezei \cite{icsoft22} разработали библиотеку для компиляторов на основе запросов и использовали её для реализации компилятора и языкового сервера для простого языка программирования.

Инкрементальный синтаксический анализ \cite{Ghezzi1979, diekmann2019editing, Wagner1998}, который обновляет синтаксическое дерево с изменениями исходного кода, также значительно ускоряет интерактивные компиляторы. Многие генераторы парсеров, такие как Tree-sitter \cite{tree-sitter} и Ohm \cite{Dubroy2017}, уже включают эту технологию.
