\chapter{Обзор литературы}
\label{chap:lr}

Это исследование основывается на работах в областях ассистентов доказательств, языковых серверов и архитектуры компиляторов на основе запросов.

\section{Ассистенты доказательств}

\Rzk{}~\cite{kudasov2023experimental} использует разработки других ассистентов для синтетических $\infty$-категорий.

\subsection{Coq и Cubical Agda}

Coq \cite{huet1997coq} поддерживает зависимые типы и индуктивные конструкции, используется в формальной верификации. UniMath \cite{DanielGrayson2024, MacPherson2019} добавляет поддержку Homotopy Type Theory. Cubical Agda \cite{VEZZOSI2021} расширяет Agda для формализации Homotopy Type Theory \cite{Cohen2016}.

\subsection{Proof General и A.L.G.A.E.}

Proof General \cite{Aspinall2000} — инструмент для Emacs, поддерживающий Coq и Isabelle. A.L.G.A.E. \footnote{\url{https://redprl.org/\#algae}} предоставляет библиотеки для разработки ассистентов на OCaml \cite{Kovacs2021}.

\section{Протокол языкового сервера}

LSP от Microsoft отделяет реализацию языка от интерфейса редактора \cite{Buender2019} и используется такими инструментами, как Visual Studio Code и Emacs. Jonas Rask et al. \cite{JonasKjaerRask2021} предложили расширение LSP для спецификационных языков.

\subsection{Lean 4 и другие отчеты}

Lean 4 использует расширение VS Code для взаимодействия с сервером Lean \cite{Nawrocki2023}. Bour et al. \cite{Bour2018} описывают сервер для OCaml, Калисзык \cite{Kaliszyk2007} предлагает веб-интерфейсы для Coq.

\section{Архитектура компиляторов на основе запросов}

Архитектура компиляторов на основе запросов \cite{ollef-rock} ускоряет интерактивные среды программирования. Ленкефи и Мезеи \cite{icsoft22} разработали библиотеку для таких компиляторов.
