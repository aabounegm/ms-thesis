\chapter{Системные требования и проектирование}
\label{chap:req}

\section{Требования}

Требования к ассистенту доказательств \Rzk{} и его расширению для VS Code делятся на основные функции и функции для удобства пользователей. Основные функции включают инкрементальную типизацию для проверки только изменённых частей кода, кэширование результатов и поддержку различных интерфейсов (CLI и языкового сервера).

\subsection{Основные функции}

Ассистент доказательств \Rzk{} должен поддерживать инкрементальную типизацию для ускорения процесса проверки, как в TypeScript \cite{Vanderkam2024}. Интерфейс должен кэшировать результаты типизации в памяти или на диске и быть гибким для различных пользовательских интерфейсов, включая CLI и языковой сервер.

\subsection{Функции для удобства пользователей}

Основные функции должны обеспечивать простой и интуитивно понятный интерфейс для новичков, быструю обратную связь, диагностические сообщения, автозавершение, информацию при наведении и переход к определению переменной. Также ожидаются функции, общие для популярных IDE, такие как интеграция с git и инструменты рефакторинга.

\section{Проектирование}

Ассистент доказательств должен поддерживать языковой сервер, реализованный с использованием LSP для интеграции с различными редакторами. Ядро ассистента предоставляет интерфейс для языкового сервера и CLI для неинтерактивного использования.

Языковой сервер, написанный на Haskell, легко интегрируется с ядром ассистента доказательств и обеспечивает функции, такие как автозавершение и диагностические сообщения.

Расширение для VS Code упрощает загрузку и использование языкового сервера, проверяя страницу релизов на GitHub\footnote{\url{https://github.com/rzk-lang/rzk/releases}} и загружая последнюю версию.

Дополнительно разработаны вспомогательные инструменты, такие как плагин для MkDocs и GitHub Action для упрощения использования \Rzk{} в проектах и CI-пайплайнах.
