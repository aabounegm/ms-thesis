\chapter{Системные требования и проектирование}
\label{chap:req}

\section{Требования}

Требования к ассистенту доказательств (и его расширению для VS Code) можно разделить на две категории: основные функции и функции для удобства пользователей. Основные функции — это необходимые функциональности и свойства, которые ассистент доказательств должен иметь для поддержки языкового сервера без особых трудностей. Функции для удобства пользователей не влияют на внутренний дизайн ассистента доказательств, но помогают обеспечить приятный пользовательский опыт.

\subsection{Основные функции}

Так как сам ассистент доказательств \Rzk{} уже находится в разработке и имеет работающий типизатор, основное внимание уделяется функциям, необходимым для использования ассистента доказательств в интерактивной среде. До сих пор ассистент доказательств имел только командный интерфейс (CLI), который проверял все входные файлы каждый раз, когда пользователь хотел проверить доказательство. Было очевидно, что это неустойчивый способ разработки доказательств, особенно для больших проектов.

К счастью, была четкая разделение между основным алгоритмом типизации и интерфейсом в виде библиотеки, что упростило работу по модификации только интерфейса без изменения алгоритма. Этот интерфейс должен поддерживать инкрементальную типизацию, что позволяет проверять только измененные части кода, значительно ускоряя процесс. Это похоже на функцию инкрементальной компиляции TypeScript, которая кэширует информацию во время каждой компиляции и компилирует только измененные файлы \cite{Vanderkam2024}.

Интерфейс должен уметь кэшировать результаты типизации в памяти и обновлять их по мере необходимости. Модуль кэширования должен поддерживать различные методы кэширования, такие как кэширование в памяти или на диске, поскольку могут быть разные пользовательские интерфейсы с разными требованиями (например, CLI против языкового сервера).

Возможность поддержки различных пользовательских интерфейсов важна, поскольку ассистент доказательств не ограничивается только VS Code. Потребителями ассистента доказательств могут быть не только люди, предпочитающие графический интерфейс, но и другие программные инструменты, такие как CI-пайплайны или другие вспомогательные инструменты. Поэтому также необходим командный интерфейс, позволяющий использовать ассистент доказательств в неинтерактивных условиях.

Наконец, интерфейс ассистента доказательств должен быть спроектирован таким образом, чтобы его легко можно было расширять и модифицировать, что обеспечивает короткий цикл разработки новых функций. Это особенно важно для ассистента доказательств, который все еще находится в разработке и имеет множество запланированных функций.

\subsection{Функции для удобства пользователей}

Основная цель описанных выше основных функций — помочь улучшить пользовательский опыт ассистента доказательств. Эти требования к UX не обязательно являются функциями или функциональностями, но также включают свойства, которые должен иметь ассистент доказательств.

Самое важное из этих свойств — простота использования для новичков (или непрограммистов) и интуитивно понятный интерфейс. Это критично для широкого распространения ассистента доказательств, так как целевая аудитория не обязательно опытные программисты. В любом случае, процесс редактирования должен быть удобным для любого пользователя.

Быстрая обратная связь по введенным данным — еще одно важное свойство, которое должен иметь ассистент доказательств. Это включает, например, диагностические сообщения, отображаемые сразу после ввода ошибки пользователем, а также информацию при наведении курсора, показывающую тип переменной и местоположение ее определения. Также включены автозавершение и предложения кода, которые помогают предотвратить использование неопределенных переменных. Пользователи также ожидают возможности перехода к определению переменной или поиска всех ссылок на нее.

Наконец, ожидаются функции, общие для большинства популярных IDE, такие как интеграция с git, отладка и инструменты рефакторинга.

\section{Проектирование}

С учетом вышеупомянутых требований ясно, что ассистент доказательств должен поддерживать языковой сервер. Кроме того, языковой сервер должен быть реализован с использованием LSP, чтобы обеспечить легкую интеграцию с различными редакторами без необходимости писать отдельное расширение или плагин для каждого текстового редактора. Ядро ассистента доказательств должно предоставлять интерфейс, который языковой сервер может использовать для взаимодействия с ним, а также CLI для неинтерактивного использования.

Сам языковой сервер будет реализован на Haskell, так как ассистент доказательств также написан на Haskell. Это позволит легко интегрировать его с ядром ассистента доказательств и упростит обслуживание кода.

Система может быть разделена на следующие компоненты: основной типизатор, CLI, языковой сервер, расширение для VS Code и утилиты. Основной алгоритм типизации — это сердце ассистента доказательств, отвечающее за проверку корректности доказательств; он выходит за рамки данного проекта, который предполагает его готовность и строится на его основе.

Затем идет интерфейс типизатора, который можно разделить на две части: CLI и языковой сервер. CLI — это простой интерфейс, позволяющий пользователю проверять файл или проект из командной строки. Он также полезен для неинтерактивного использования, такого как CI-пайплайны или другие инструменты. Его простота также делает его разумным интерфейсом для тестирования новых функций, так как здесь меньше подвижных частей.

Другой интерфейс — это языковой сервер, который отвечает за предоставление интерактивного интерфейса к ассистенту доказательств. Это более удобный из двух интерфейсов и будет использоваться большинством пользователей. Он реализован с использованием протокола Language Server от Microsoft для обеспечения легкой интеграции с различными текстовыми редакторами, такими как VS Code, Vim или Emacs. Языковой сервер также отвечает за предоставление функций, ожидаемых от современного языка программирования, таких как автозавершение, диагностические сообщения и информация при наведении.

Расширение для VS Code — это тонкая оболочка вокруг языкового сервера, которая помогает легко загружать копию языкового сервера и интегрировать его с VS Code. Оно делает это, проверяя страницу релизов в репозитории GitHub\footnote{\url{https://github.com/rzk-lang/rzk/releases}} и загружая последнюю соответствующую бинарную версию для операционной системы пользователя, которая затем кэшируется в локальном каталоге хранения расширения, предоставляемом VS Code. Расширение также упрощает сборку языкового сервера из исходников, если пользователь предпочитает это сделать. Дополнительно, оно предоставляет опцию указания пользовательского пути к языковому серверу для облегчения тестирования и разработки.

Расширение для VS Code доступно на нескольких платформах, включая Visual Studio Marketplace, Open VSX и GitHub. Это делается для того, чтобы расширение было легко обнаружимо пользователями и могло быть установлено из разных источников, особенно для пользователей, которые предпочитают не использовать проприетарное ПО Microsoft и вместо этого выбирают альтернативы, такие как VSCodium\footnote{\url{https://vscodium.com}}. VSCodium можно настроить для использования Open VSX в качестве своего рынка расширений, что позволяет пользователям устанавливать расширение оттуда. Наличие файла ".vsix" на странице релизов репозитория GitHub также способствует открытости и гибкости проекта и позволяет пользователям устанавливать расширение вручную, если они предпочитают это делать. Выпуск на все эти платформы автоматизирован с помощью GitHub Actions, который собирает расширение и загружает его на соответствующие платформы.

Наконец, разработаны вспомогательные инструменты для выполнения мелких задач, связанных с использованием ассистента доказательств. Например, есть плагин Python\footnote{\url{https://pypi.org/project/mkdocs-plugin-rzk/}} для MkDocs, который автоматизирует рендеринг SVG файлов в проекте формализации, использующем MkDocs, что полезно для рендеринга топов в результатирующей документации. Другой инструмент — это GitHub Action\footnote{\url{https://github.com/rzk-lang/rzk-action}}, упрощающий процесс установки \Rzk{} в CI-пайплайне и запуска его на файлах проекта.
