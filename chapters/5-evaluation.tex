\chapter{Evaluation and Discussion}
\label{chap:eval}

\section{Evaluation}

The work on the \Rzk{} proof assistant and its language server has been evaluated
by getting feedback directly from users and by comparing the features of the proof assistant
to other proof assistants in the field.
Overall, the feedback has been positive, with users appreciating the new features
and the improved user experience.
While the language server is still in development and lacks some features that are present in
many other language servers, it covers the basic requirements and has been shown to be useful in practice.
It implemented features such as incremental typechecking, semantic highlighting, diagnostics, formatting,
and code completions, which are important for the editing experience of any programming language.
The current implementation also sets the base for future work and more language features,
which now have a clear path to be implemented.

\subsection{ITP Summer School 2023}

The main form of user feedback collection was during the 2-week summer school on
Interaction of Proof Assistants and Mathematics\footnote{\url{https://itp-school-2023.github.io/}}
that took place in September 2023 in Regensburg, Germany.
A tutorial on \Rzk{} and how to use it was given to the participants and they were
observed while using the proof assistant and the language server.
The feedback was collected through informal discussions with the participants,
and the requested features or fixes were noted down or worked on immediately.
This also included feedback by the proponents of the type theory on which \Rzk{} is based.

\subsection{Formalization projects}

The \Rzk{} proof assistant has been used in several community-driven formalization projects.
These projects have been used to evaluate the usability of the tools surrounding the
proof assistant, including the language server and the VS Code extension.

The most notable of these projects include:
\begin{itemize}
  \item \textbf{sHoTT}\footnote{\url{https://github.com/rzk-lang/sHoTT}}:
        an active formalisation project for simplicial homotopy type theory and $\infty$-categories.
  \item \textbf{HoTT Book}\footnote{\url{https://github.com/rzk-lang/hottbook}}:
        formalization of the theorems and exercises from the Homotopy Type Theory book~\cite{hottbook}.
  \item \textbf{Yoneda}\footnote{\url{https://github.com/emilyriehl/yoneda}}:
        comparative formalizations of the Yoneda lemma for 1-categories and $\infty$-categories.
\end{itemize}

\section{Discussion and reflection}

The results of this work have proven to be useful to mathematicians and
computer scientists studying higher category theory and homotopy type theory.
While the user base of the proof assistant is not large (and is not expected to be),
the editing experience of those who use it has been significantly improved.
Further improvements can be achieved by employing automated testing tools and
by conducting a formal user study to evaluate the effectiveness of the new features.
More LSP feature should additionally be implemented for an even smoother \Rzk{} editing experience.
