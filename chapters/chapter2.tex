\chapter{Literature Review}
\label{chap:lr}
\chaptermark{Second Chapter Heading}

\section{Language Server Protocol}

The Language Server Protocol is a common protocol for programming language analyzers to communicate with development tools. The Language Server Protocol is used between a tool (the client) and a language smartness provider (the server) to integrate features like auto complete, go to definition, find all references and alike into the tool. The Language Server Protocol is used by many development tools like Visual Studio Code, Eclipse, Emacs, Sublime Text, Atom, and Vim. The Language Server Protocol is developed by Microsoft and is released under the Creative Commons Attribution License.

Its development started on June 27, 2016.
Microsoft collaborated with Red Hat and Codenvy to standardize the protocol's specifictation.
% https://www.infoworld.com/article/3088698/microsoft-backed-langauge-server-protocol-strives-for-language-tools-interoperability.html
% https://sdtimes.com/che/codenvy-microsoft-red-hat-collaborate-language-server-protocol/

\subsection{SLSP}

LSP is not good enough for "specification languages" (SL) because it is not designed for them. SLSP is a protocol for specification languages based on LSP.

It adds missing features (requests and notification types) for specification languages. It also adds a new feature for specification languages: "proof view" (a view of the proof state - like Lean's Info View).

% TODO: figure out what it uses DAP for exactly.

\section{Proof Assistants}
