\documentclass[oneside,final,14pt,a4paper]{extreport}

\usepackage{tempora}

\usepackage{vmargin}
\setpapersize{A4}
\setmarginsrb{2.5cm}{2.2cm}{2.2cm}{2.2cm}{0pt}{10mm}{0pt}{13mm}
\usepackage{setspace}
\sloppy
\setstretch{1.5}
\usepackage{indentfirst}
\parindent=1.25cm

%%%%% ADDED TO SUPPORT TT BOLD FACES %%%%
\DeclareFontShape{OT1}{cmtt}{bx}{n}{<5><6><7><8><9><10><10.95><12><14.4><17.28><20.74><24.88>cmttb10}{}
\renewcommand{\ttdefault}{pcr}
%%%%% END %%%%%%%%%%%%%%%%%%%%%%%%%%%%%%%
\usepackage{atbegshi,picture}
\usepackage[T1,T2A]{fontenc}
\usepackage[utf8]{inputenc}
\usepackage{fontspec}
\setmainfont{Times New Roman}
\usepackage[main=russian,english]{babel}
\usepackage[backend=biber,style=ieee,autocite=inline]{biblatex}
\bibliography{ref.bib}
\usepackage{csquotes}
\usepackage{blindtext}


\usepackage{pdfpages}
\newenvironment{bottompar}{\par\vspace*{\fill}}{\clearpage}

% \usepackage{cite}
\usepackage{amsmath,amsfonts}
\usepackage{amsthm}
\newtheorem{theorem}{Theorem}
\newtheorem{corollary}{Corollary}
\newtheorem{lemma}{Lemma}
\newtheorem{proposition}{Proposition}
\theoremstyle{definition}
\newtheorem{definition}{Definition}
\theoremstyle{remark}
\newtheorem*{remark}{Remark}
\theoremstyle{remark}
\newtheorem*{example}{Example}



\usepackage{graphicx}
\graphicspath{{figs/}} %path to images
\usepackage{multirow,array}
\usepackage{caption}
\usepackage{subcaption}
\usepackage[unicode]{hyperref}
\hypersetup{colorlinks=true, linkcolor=black, citecolor=black}
\usepackage{paralist}
\usepackage{listings}
\usepackage{zed-csp}
\usepackage{fancyhdr}
\usepackage{color,colortbl}
\usepackage{booktabs}
\usepackage{epsfig} % for postscript graphics files

\usepackage{upgreek}
\usepackage{bm}
\usepackage{hyperref}
\usepackage{longtable}
\usepackage[font=singlespacing, labelfont=bf]{caption}
\usepackage{floatrow}

\pagestyle{fancyplain}

% remember section title
\renewcommand{\chaptermark}[1]%
	{\markboth{\chaptername~\thechapter~--~#1}{}}

% subsection number and title
\renewcommand{\sectionmark}[1]%
	{\markright{\thesection\ #1}}

\rhead[\fancyplain{}{\bf\leftmark}]%
      {\fancyplain{}{\bf\thepage}}
\lhead[\fancyplain{}{\bf\thepage}]%
      {\fancyplain{}{\bf\rightmark}}
\cfoot{} %bfseries


\newcommand{\dedication}[1]
   {\thispagestyle{empty}

   \begin{flushleft}\raggedleft #1\end{flushleft}
}

\newcommand{\Rzk}{\textsc{Rzk}}

\begin{document}

\includepdf[pages=-,offset={-\hoffset} {\voffset}]{annotation-title.pdf}

\newpage
\tableofcontents
\begin{abstract}
  % skip one line to make the abstract start with indent

  Теория типов Рила-Шулмана для синтетических $\infty$-категорий - это новая теория, основанная на теории типов гомотопий (HoTT).
  Экспериментальный помощник доказательств \Rzk{} предлагает автоматическую проверку доказательств для этой теории.
  С целью сделать эту теорию более доступной для математиков и компьютерных ученых,
  мы представляем работу по созданию набора командных и интерактивных инструментов для \Rzk{}.
  Эти инструменты включают сервер языка, расширение для Visual Studio Code (VS Code),
  и список небольших утилит для удобства.
  Хотя мы сосредоточены на поддержке VS Code,
  сервер языка также совместим с другими популярными редакторами, поддерживающими Language Server Protocol (LSP),
  такими как Emacs и Vim.
\end{abstract}

\setcounter{page}{4}
% set manually the number, from which Глава 1 starts!
% Why do we put 4 in this case?
% Title page - page 1
% Оглавление - page 2
% Аннотация - page 3
% Глава 1 - page 4
% In your annotation the counter number can be different, please count carefully and insert the corresponding number.

\chapter{Введение}
\label{chap:intro}

\section{Ассистенты доказательств}

Ассистенты доказательств, также известные как интерактивные доказательные системы (ITP),
это программные инструменты, используемые в математике, компьютерных науках и формальных методах
для помощи в разработке и проверке математических доказательств.
Эти инструменты играют важную роль в обеспечении корректности и надежности
сложных математических утверждений и программных систем.
Примеры таких инструментов включают Agda~\cite{BoveDybjerNorell2009}, Coq~\cite{BertotCasteran2013} и Lean~\cite{deMouraUllrich2021}.

Хотя проверка доказательств похожа на типизацию в языках программирования,
она обычно рассматривается как интерактивный процесс между математиком и ассистентом доказательств.
Для этого большинство ассистентов доказательств предоставляют интерактивные возможности через специализированные IDE
(например, CoqIDE\footnote{\url{https://coq.inria.fr/refman/practical-tools/coqide.html}}),
интеграцию с Proof General~\cite{Aspinall2000}, или Visual Studio Code
(через Language Server Protocol (LSP)~\cite{Gunasinghe2022}).
Большинство известных ассистентов доказательств поддерживают несколько таких опций.

\Rzk{}~\cite{Kudasov2023-github-rzk} - новый ассистент доказательств, основанный на теории типов Рила-Шулмана для синтетических $\infty$-категорий~\cite{RiehlShulman2017, Riehl2023}.
Этот экспериментальный ассистент недавно был успешно использован для формализации некоторых фундаментальных результатов для $\infty$-категорий,
включая лемму Йонеды для $\infty$-категорий~\cite{Kudasov2023}.
Однако \Rzk{} не имел многих из упомянутых интерактивных функций, таких как подсветка синтаксиса.
В этой статье мы описываем реализацию утилит и интерактивных инструментов для \Rzk{},
с акцентом на поддержку сервера языка и расширения для VS Code.
Цель - предоставить более удобный интерфейс для \Rzk{}, который будет полезен как новичкам, так и опытным пользователям.

\section{Серверы языков}

Несколько лет назад, когда появлялся новый редактор кода,
он должен был поддерживать популярные языки программирования
и зависел от плагинов, написанных специально для этого редактора, чтобы поддерживать другие языки.
Это приводило к множеству повторяющихся работ для обеспечения функций языка в разных редакторах
и к несоответствиям между функциями языка в разных редакторах.
Идея серверов языков, предложенная Microsoft, решает эту проблему,
стандартизируя протокол для общения между редактором кода (клиентом) и фоновым процессом (сервером),
предоставляющим функции языка для определенного языка. Этот протокол известен как Language Server Protocol (LSP) \cite{Gunasinghe2022}.
С LSP разработчику языка нужно реализовать сервер один раз,
и он автоматически получит поддержку языка в любом редакторе, поддерживающем LSP.
Точно так же редактор, поддерживающий LSP, автоматически получит поддержку любого языка программирования, имеющего LSP-сервер.
Примеры функций языка включают предоставление диагностических сообщений, переход к месту определения идентификатора, автозаполнение текста, семантическую подсветку синтаксиса и многое другое.

\begin{figure}
  \centering
  \includegraphics[width=0.7\textwidth]{figs/LSP-MxN.png}
  \label{figure:lsp}
  \caption{
    Мотивация создания LSP, с сайта Microsoft.
    \protect\footnotemark
  }
\end{figure}
\footnotetext{\raggedright\url{https://code.visualstudio.com/api/language-extensions/language-server-extension-guide}}

Этот протокол особенно полезен для интерактивных доказательных систем, поскольку
они больше зависят от удобства редактирования, чем от командной строки.
Это связано с тем, что доказательные системы не нуждаются в компиляции
в исполняемый файл и требуют только этапа проверки типов компиляторов,
который легко может быть выполнен сервером языка. Однако это также увеличивает нагрузку
на сервер языка, так как он должен поддерживать больше функций, чем обычный компилятор,
включая отображение переменных в контексте с их типами, поддержку символов Unicode
и, самое главное, возможность пошагового интерактивного доказательства.

\section{Вклад}

В этой статье мы описываем работу по реализации утилит и интерактивных инструментов для \Rzk{},
с акцентом на поддержку сервера языка и расширения для VS Code.
Ранее отчет о прогрессе этой работы был представлен на конференции PSSV 2023
в Иннополисском университете \cite{PSSV2023}.

Основные вклады включают следующее:
\begin{itemize}
  \item Сервер языка для \Rzk{}, поддерживающий основные функции LSP, такие как диагностика, автозаполнение кода и семантическая подсветка синтаксиса.
  \item Расширение для VS Code, позволяющее пользователям легко загружать и использовать сервер языка.
  \item Форматтер кода для \Rzk{}.
  \item Плагин для MkDocs, позволяющий рендерить фрагменты кода \Rzk{} в документах Markdown.
  \item Интеграция упомянутых дополнений в существующие проекты формализации \Rzk{}.
\end{itemize}

\chapter{Обзор литературы}
\label{chap:lr}

Это исследование основывается на работах в областях ассистентов доказательств, языковых серверов и архитектуры компиляторов на основе запросов.

\section{Ассистенты доказательств}

\Rzk{}~\cite{kudasov2023experimental} использует разработки других ассистентов для синтетических $\infty$-категорий.

\subsection{Coq и Cubical Agda}

Coq \cite{huet1997coq} поддерживает зависимые типы и индуктивные конструкции, используется в формальной верификации. UniMath \cite{DanielGrayson2024, MacPherson2019} добавляет поддержку Homotopy Type Theory. Cubical Agda \cite{VEZZOSI2021} расширяет Agda для формализации Homotopy Type Theory \cite{Cohen2016}.

\subsection{Proof General и A.L.G.A.E.}

Proof General \cite{Aspinall2000} — инструмент для Emacs, поддерживающий Coq и Isabelle. A.L.G.A.E. \footnote{\url{https://redprl.org/\#algae}} предоставляет библиотеки для разработки ассистентов на OCaml \cite{Kovacs2021}.

\section{Протокол языкового сервера}

LSP от Microsoft отделяет реализацию языка от интерфейса редактора \cite{Buender2019} и используется такими инструментами, как Visual Studio Code и Emacs. Jonas Rask et al. \cite{JonasKjaerRask2021} предложили расширение LSP для спецификационных языков.

\subsection{Lean 4 и другие отчеты}

Lean 4 использует расширение VS Code для взаимодействия с сервером Lean \cite{Nawrocki2023}. Bour et al. \cite{Bour2018} описывают сервер для OCaml, Калисзык \cite{Kaliszyk2007} предлагает веб-интерфейсы для Coq.

\section{Архитектура компиляторов на основе запросов}

Архитектура компиляторов на основе запросов \cite{ollef-rock} ускоряет интерактивные среды программирования. Ленкефи и Мезеи \cite{icsoft22} разработали библиотеку для таких компиляторов.

\chapter{Системные требования и проектирование}
\label{chap:req}

\section{Требования}

Требования к ассистенту доказательств (и его расширению для VS Code) можно разделить на две категории: основные функции и функции для удобства пользователей. Основные функции — это необходимые функциональности и свойства, которые ассистент доказательств должен иметь для поддержки языкового сервера без особых трудностей. Функции для удобства пользователей не влияют на внутренний дизайн ассистента доказательств, но помогают обеспечить приятный пользовательский опыт.

\subsection{Основные функции}

Так как сам ассистент доказательств \Rzk{} уже находится в разработке и имеет работающий типизатор, основное внимание уделяется функциям, необходимым для использования ассистента доказательств в интерактивной среде. До сих пор ассистент доказательств имел только командный интерфейс (CLI), который проверял все входные файлы каждый раз, когда пользователь хотел проверить доказательство. Было очевидно, что это неустойчивый способ разработки доказательств, особенно для больших проектов.

К счастью, была четкая разделение между основным алгоритмом типизации и интерфейсом в виде библиотеки, что упростило работу по модификации только интерфейса без изменения алгоритма. Этот интерфейс должен поддерживать инкрементальную типизацию, что позволяет проверять только измененные части кода, значительно ускоряя процесс. Это похоже на функцию инкрементальной компиляции TypeScript, которая кэширует информацию во время каждой компиляции и компилирует только измененные файлы \cite{Vanderkam2024}.

Интерфейс должен уметь кэшировать результаты типизации в памяти и обновлять их по мере необходимости. Модуль кэширования должен поддерживать различные методы кэширования, такие как кэширование в памяти или на диске, поскольку могут быть разные пользовательские интерфейсы с разными требованиями (например, CLI против языкового сервера).

Возможность поддержки различных пользовательских интерфейсов важна, поскольку ассистент доказательств не ограничивается только VS Code. Потребителями ассистента доказательств могут быть не только люди, предпочитающие графический интерфейс, но и другие программные инструменты, такие как CI-пайплайны или другие вспомогательные инструменты. Поэтому также необходим командный интерфейс, позволяющий использовать ассистент доказательств в неинтерактивных условиях.

Наконец, интерфейс ассистента доказательств должен быть спроектирован таким образом, чтобы его легко можно было расширять и модифицировать, что обеспечивает короткий цикл разработки новых функций. Это особенно важно для ассистента доказательств, который все еще находится в разработке и имеет множество запланированных функций.

\subsection{Функции для удобства пользователей}

Основная цель описанных выше основных функций — помочь улучшить пользовательский опыт ассистента доказательств. Эти требования к UX не обязательно являются функциями или функциональностями, но также включают свойства, которые должен иметь ассистент доказательств.

Самое важное из этих свойств — простота использования для новичков (или непрограммистов) и интуитивно понятный интерфейс. Это критично для широкого распространения ассистента доказательств, так как целевая аудитория не обязательно опытные программисты. В любом случае, процесс редактирования должен быть удобным для любого пользователя.

Быстрая обратная связь по введенным данным — еще одно важное свойство, которое должен иметь ассистент доказательств. Это включает, например, диагностические сообщения, отображаемые сразу после ввода ошибки пользователем, а также информацию при наведении курсора, показывающую тип переменной и местоположение ее определения. Также включены автозавершение и предложения кода, которые помогают предотвратить использование неопределенных переменных. Пользователи также ожидают возможности перехода к определению переменной или поиска всех ссылок на нее.

Наконец, ожидаются функции, общие для большинства популярных IDE, такие как интеграция с git, отладка и инструменты рефакторинга.

\section{Проектирование}

С учетом вышеупомянутых требований ясно, что ассистент доказательств должен поддерживать языковой сервер. Кроме того, языковой сервер должен быть реализован с использованием LSP, чтобы обеспечить легкую интеграцию с различными редакторами без необходимости писать отдельное расширение или плагин для каждого текстового редактора. Ядро ассистента доказательств должно предоставлять интерфейс, который языковой сервер может использовать для взаимодействия с ним, а также CLI для неинтерактивного использования.

Сам языковой сервер будет реализован на Haskell, так как ассистент доказательств также написан на Haskell. Это позволит легко интегрировать его с ядром ассистента доказательств и упростит обслуживание кода.

Система может быть разделена на следующие компоненты: основной типизатор, CLI, языковой сервер, расширение для VS Code и утилиты. Основной алгоритм типизации — это сердце ассистента доказательств, отвечающее за проверку корректности доказательств; он выходит за рамки данного проекта, который предполагает его готовность и строится на его основе.

Затем идет интерфейс типизатора, который можно разделить на две части: CLI и языковой сервер. CLI — это простой интерфейс, позволяющий пользователю проверять файл или проект из командной строки. Он также полезен для неинтерактивного использования, такого как CI-пайплайны или другие инструменты. Его простота также делает его разумным интерфейсом для тестирования новых функций, так как здесь меньше подвижных частей.

Другой интерфейс — это языковой сервер, который отвечает за предоставление интерактивного интерфейса к ассистенту доказательств. Это более удобный из двух интерфейсов и будет использоваться большинством пользователей. Он реализован с использованием протокола Language Server от Microsoft для обеспечения легкой интеграции с различными текстовыми редакторами, такими как VS Code, Vim или Emacs. Языковой сервер также отвечает за предоставление функций, ожидаемых от современного языка программирования, таких как автозавершение, диагностические сообщения и информация при наведении.

Расширение для VS Code — это тонкая оболочка вокруг языкового сервера, которая помогает легко загружать копию языкового сервера и интегрировать его с VS Code. Оно делает это, проверяя страницу релизов в репозитории GitHub\footnote{\url{https://github.com/rzk-lang/rzk/releases}} и загружая последнюю соответствующую бинарную версию для операционной системы пользователя, которая затем кэшируется в локальном каталоге хранения расширения, предоставляемом VS Code. Расширение также упрощает сборку языкового сервера из исходников, если пользователь предпочитает это сделать. Дополнительно, оно предоставляет опцию указания пользовательского пути к языковому серверу для облегчения тестирования и разработки.

Расширение для VS Code доступно на нескольких платформах, включая Visual Studio Marketplace, Open VSX и GitHub. Это делается для того, чтобы расширение было легко обнаружимо пользователями и могло быть установлено из разных источников, особенно для пользователей, которые предпочитают не использовать проприетарное ПО Microsoft и вместо этого выбирают альтернативы, такие как VSCodium\footnote{\url{https://vscodium.com}}. VSCodium можно настроить для использования Open VSX в качестве своего рынка расширений, что позволяет пользователям устанавливать расширение оттуда. Наличие файла ".vsix" на странице релизов репозитория GitHub также способствует открытости и гибкости проекта и позволяет пользователям устанавливать расширение вручную, если они предпочитают это делать. Выпуск на все эти платформы автоматизирован с помощью GitHub Actions, который собирает расширение и загружает его на соответствующие платформы.

Наконец, разработаны вспомогательные инструменты для выполнения мелких задач, связанных с использованием ассистента доказательств. Например, есть плагин Python\footnote{\url{https://pypi.org/project/mkdocs-plugin-rzk/}} для MkDocs, который автоматизирует рендеринг SVG файлов в проекте формализации, использующем MkDocs, что полезно для рендеринга топов в результатирующей документации. Другой инструмент — это GitHub Action\footnote{\url{https://github.com/rzk-lang/rzk-action}}, упрощающий процесс установки \Rzk{} в CI-пайплайне и запуска его на файлах проекта.

\chapter{Реализация}
\label{chap:impl}

\section{Языковой сервер \Rzk{} и расширение для VS Code}

Описываем реализацию языкового сервера и расширения для \Rzk{} в среде VS Code. Языковой сервер имеет доступ к внутренним компонентам доказательного ассистента и предоставляет интерфейс, соответствующий протоколу языкового сервера. Расширение для VS Code действует как посредник между редактором и языковым сервером.

\subsection{Особенности}

Языковой сервер поддерживает интуитивный интерфейс и подсветку синтаксиса, соответствующую ожиданиям математиков и программистов. Пользователи получают ясную навигацию и подсветку синтаксиса, что улучшает читаемость кода и облегчает обнаружение ошибок.

\subsubsection{Автодополнение кода и предложения}

Расширение для VS Code \Rzk{} использует LSP для автодополнения кода и предложений, помогая писать код более эффективно и снижая вероятность ошибок.

\subsection{Расширение для VS Code}

Расширение управляет установкой языкового сервера на основных ОС, используя бинарные файлы с GitHub. Исходный код доступен на GitHub\footnote{\url{https://github.com/rzk-lang/vscode-rzk}}, а расширение на Visual Studio Marketplace\footnote{\url{https://marketplace.visualstudio.com/items?itemName=NikolaiKudasovfizruk.rzk-1-experimental-highlighting}} и Open VSX\footnote{\url{https://open-vsx.org/extension/NikolaiKudasovfizruk/rzk-1-experimental-highlighting}}.

\subsubsection{Установка и активация}

После активации расширение проверяет наличие \Rzk{} в \texttt{PATH}. Если исполняемый файл не найден, предлагается скачать его с GitHub. Пользователь может указать пользовательский путь к бинарному файлу.

\subsection{Языковой сервер \Rzk{}}

Языковой сервер, написанный на Haskell, поддерживает подсветку синтаксиса, диагностические сообщения и автодополнение текста. Он совместим с любыми редакторами, поддерживающими LSP.

\section{Вспомогательные инструменты}

Разработаны вспомогательные инструменты, такие как плагин для MkDocs и GitHub Action для использования \Rzk{} в CI-пайплайнах.

\subsection{Плагин MkDocs}

Плагин для MkDocs, доступный на PyPI\footnote{\url{https://pypi.org/project/mkdocs-plugin-rzk/}}, позволяет рендерить диаграммы в документации, используя \Rzk{}.

\subsection{GitHub Action}

GitHub Action, доступный на GitHub\footnote{\url{https://github.com/rzk-lang/rzk-action}}, позволяет проверять корректность доказательств в CI-пайплайнах, обеспечивая статус проверки в pull requests.

\chapter{Оценка и обсуждение}
\label{chap:eval}

\section{Оценка}

Обратная связь по \Rzk{} и его языковому серверу была собрана через прямое взаимодействие с пользователями и сравнение с другими доказательными помощниками. Пользователи выразили удовлетворение новыми функциями и улучшенным пользовательским опытом. Хотя сервер развивается и не имеет некоторых функций, присутствующих в других серверах, он удовлетворяет базовым требованиям и полезен на практике.

\subsection{Летняя школа ITP 2023}

Основной отзывы пользователей были собраны на летней школе по взаимодействию доказательных помощников и математики в 2023 году в Регенсбурге, Германия. Участники получили обучение по \Rzk{}, и отзывы были собраны через неформальные обсуждения. Запросы на функции и исправления были замечены и решены быстро, включая отзывы от экспертов по теории типов.

\subsection{Проекты формализации}

\Rzk{} был применен в нескольких сообщественных проектах по формализации, оценивающих удобство сопутствующих инструментов, таких как языковой сервер и расширение для VS Code. Заметные проекты включают \textbf{sHoTT}, \textbf{HoTT Book} и \textbf{Yoneda}, фокусирующиеся на симплициальной гомотопической теории типов, теоремах гомотопической теории типов и сравнительных формализациях леммы Йонеды.

\section{Обсуждение и рефлексия}

\Rzk{} поддерживает математиков и компьютерных ученых, изучающих высшую категориальную теорию и гомотопическую теорию типов. Несмотря на скромное количество пользователей, отзывы показывают значительное улучшение опыта редактирования. В будущем планируется внедрение автоматизированных инструментов тестирования, формальных исследований пользователей и дополнительных функций LSP для более гладкого редактирования.

\chapter{Заключение}
\label{chap:conclusion}

Мы разработали и реализовали начальный прототип языкового сервера и расширения для VS Code для экспериментального доказательного помощника \Rzk{}. Промежуточные результаты были представлены на школе \textit{Взаимодействие доказательных помощников и математики} в Германии в сентябре 2023 года, что помогло собрать обратную связь и непосредственно реагировать на неё. Текущие пользователи в основном удовлетворены прототипом, но также предоставили полезные предложения для дальнейшего улучшения.

Текущая реализация языкового сервера для \Rzk{} находится на ранних стадиях, и есть множество отсутствующих функций и мест для улучшения. В будущем планируется добавить поддержку отображения информации о переменных при наведении, переход к определению, переименование символов и другие полезные функции LSP, а также отображение топов как изображений и информационное веб-представление. Ожидается, что из этого проекта вырастет библиотека на Haskell для разработки языковых серверов, особенно для доказательных помощников.



\printbibliography[heading=bibintoc,title={Список использованной литературы}]
\end{document}
